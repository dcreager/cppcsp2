\documentclass{article}

\newcommand{\occam}{occam}
\newcommand{\occampi}{occam-$\pi$}
\newcommand{\csharp}{C\#}
\newcommand{\code}[1]{{\small\texttt{#1}}}
\newcommand{\comment}[1]{}

\begin{document}

\section{Poison Problem \#1 -- Sputh's bug}

Given:

{\small\begin{verbatim}
    One2OneChannel<int> c;
\end{verbatim}}

Consider running:

{\small\begin{verbatim}
    c.writer() << 7;
    c.writer().poison();
\end{verbatim}}

In parallel with:

{\small\begin{verbatim}
    int n;
    c.reader() >> n;
\end{verbatim}}

The latter code should never have a poison exception thrown.

\section{Poison Problem \#2 -- Destruction}

Given:

{\small\begin{verbatim}
    One2OneChannel< pair< int, Chanout<int> > > c;
\end{verbatim}}

Consider running:

{\small\begin{verbatim}
    {
        One2OneChannel<int> reply;
        int n;
        c.writer() << pair(7,reply.writer());
        reply.reader() >> n;
    }
\end{verbatim}}

In parallel with:

{\small\begin{verbatim}
    pair< int, Chanout<int> > p;
    c.reader() >> p;
    p.second << (p.first + 1);
\end{verbatim}}

Under the current code, and the `new mutex' solution, this code can crash (depending upon the scheduler, and compiler).  Under the state variable
solution, it will not crash.


\section{Original C++CSP2 Channel Code}

{\small\begin{verbatim}
    Process* volatile waiting;
    union {
        const DATA_TYPE* volatile src;
        DATA_TYPE* volatile dest;
    };
    volatile bool isPoisoned;
    volatile bool commFinished;
    Mutex mutex;

    void checkPoison() {
        if (isPoisoned) {
            mutex.release();
            throw PoisonException();
        }
    }

    void input(DATA_TYPE* paramDest) {
        mutex.claim();
        checkPoison();        

        if (waiting != NULL) {
            //Writer waiting:
            *paramDest = *src;
            Process* wasWaiting = waiting;
            waiting = NULL; src = NULL;
            commFinished = true;
            freeProcessNoAlt(wasWaiting);
            mutex.release();
        } else {
            //No-one waiting:
            dest = paramDest;
            waiting = currentProcess();
            commFinished = false;
            mutex.release();
            reschedule();
            //Now communication has finished (or we've been poisoned)
            mutex.claim();
            if (false == commFinished)
                checkPoison();
            mutex.release();
        }
    }

    void beginExtInput(DATA_TYPE* paramDest) {
        mutex.claim();
        checkPoison();

        if (waiting != NULL) {
            //Writer waiting:
            *paramDest = *src;
            mutex.release();
        } else {
            //No-one waiting:
            dest = NULL;
            waiting = currentProcess();
            commFinished = false;
            mutex.release();
            reschedule();
            //Now we are in the extended action (or we've been poisoned)
            mutex.claim();
            if (false == commFinished)
                checkPoison();             
            *paramDest = *src;
            commFinished = false;
            mutex.release();
        }
    }

    //This method will never throw a poison exception
    void endExtInput() {
        //The reader may have poisoned since beginExtInput
        //If that is the case, we do not want to throw a PoisonException to them
        //The writer can't have poisoned (they are blocked) so it must have been the reader:
        //No need to claim the mutex

        if (false == isPoisoned) {
            Process* wasWaiting = waiting;
            waiting = NULL; src = NULL;
            commFinished = true;
            freeProcessNoAlt(wasWaiting);
        }
    }

    void output(const DATA_TYPE* paramSrc) {
        mutex.claim();
        checkPoison();

        if (waiting != NULL) {
            //Reader waiting:
            if (dest != NULL) {
                //Normal reader:
                *dest = *paramSrc;
                Process* wasWaiting = waiting;
                waiting = NULL; dest = NULL;
                commFinished = true;
                freeProcessNoAlt(wasWaiting);
                mutex.release();
            } else {
                //Alting/extended reader:
                Process* wasWaiting = waiting;
                waiting = currentProcess();
                src = paramSrc;
                commFinished = true;
                freeProcessAlting(wasWaiting);
                mutex.release();
                reschedule();
                //Now communication has finished (or we've been poisoned)
                mutex.claim();
                if (false == commFinished)
                    checkPoison();
                mutex.release();
            }
        } else {
            //No-one waiting:
            src = paramSrc;
            waiting = currentProcess();
            commFinished = false;
            mutex.release();            
            reschedule();
            //Now communication has finished (or we've been poisoned)
            mutex.claim();
            if (false == commFinished)
                checkPoison();
            mutex.release();
        }
    }

    void poisonIn() {
        mutex.claim();
        isPoisoned = true;
        //Don't change commFinished
        
        Process* wasWaiting = waiting;
        waiting = NULL;

        if (wasWaiting != NULL) {
            //Might be alting, might not:
            freeProcessAlting(wasWaiting);
        }

        mutex.release();

    }

    void poisonOut() {
        poisonIn();
    }

    //Guard methods:

    bool enable(Process* proc) {
        mutex.claim();

        if (isPoisoned) {
            mutex.release();
            return true;
        } else if (waiting != NULL) {
            //Someone is ready to write
            mutex.release();
            return true;
        } else {
            //Put ourselves in the channel:
            waiting = proc;
            dest = NULL;
            mutex.release();
            return false;
        }
    }

    bool disable(Process* proc) {
        mutex.claim();

        if (isPoisoned) {
            mutex.release();
            return true;
        } else if (waiting != NULL && waiting != proc) {
            //Someone is in the channel and its not us:
            mutex.release();
            return true;
        } else {
            //Only us in the channel, remove ourselves:
            waiting = NullProcessPtr;
            mutex.release();
            return false;
        }
    }
\end{verbatim}}



\section{Proposed New Algorithm -- New Mutex}

Using this method, every time a process wakes up, it has to claim the mutex again to check whether it was poisoned or not.  This mutex claim may be 
contested, which may require a further context-switch if spinning does not allow the claim to succeed.  In section \ref{sec-alting} a JCSP method 
using (effectively) a mutex-protected shared variable was transformed into an atomically updated variable.  This could be done here -- but a better 
method is to combine the atomic variable with the mutex.

In section \ref{sec-mutex-analysis}~the spin mutex was chosen for use with channels.  As explained in section \ref{sec-mutex-spin}, spin mutexes use 
an atomic variable that is set to 0 (unclaimed) or 1 (claimed).  I propose a new special mutex that can also take values 2 (channel poisoned, last 
communication finished) and 3 (channel poisoned, last communication did not finish).  The attempt to claim the mutex is still an atomic compare and 
swap, attempting to change the value from 0 to 1.  If the value turns out to have been 2 or 3, a poison exception may be thrown appropriately.  So for 
example, the mutex will be implemented as follows:

{\small\begin{verbatim}
//class ChannelMutex
    volatile int value;
    
    bool claimNoPoison() {
        int oldVal;
        do {
            oldVal = AtomicCompareAndSwap(&value,
                /*COMPARE:*/ 0,
                /*SWAP:*/ 1);
            
            if (oldVal == 0) {
                //claimed ok - done
                return true;
            } else if (oldVal == 2 || oldVal == 3) {
                //poison!
                return false;
            } else {
                //Someone else has it
                spin();
            }            
        }
        while (oldVal == 1);
        
        return true;
    }
    
    void claimAtStart() {
        if (false == claimNoPoison())
            throw PoisonException();
    }        
    
    void checkForPoisonAfterComm() {
        int oldVal = AtomicGet(&value);
        //if (oldVal == 2)
        //    poisoned but last comm was ok - don't throw
        if (oldVal == 3)
            throw PoisonException();
        //otherwise not poisoned
    }
    
    bool checkForPoisonGeneral() {
        int oldVal = AtomicGet(&value);
        return (oldVal == 2 || oldVal == 3);
    }
    
    void release() {
        AtomicPut(&value,0);
    }
    
    void poisonDuringComm() {
        AtomicPut(&value,3);
    }
    
    void poison() {
        AtomicPut(&value,2);
    }
\end{verbatim}}

The channel algorithms will then be as follows:

{\small\begin{verbatim}
    Process* volatile waiting;
    union {
        const DATA_TYPE* volatile src;
        DATA_TYPE* volatile dest;
    };
    ChannelMutex mutex;

    void input(DATA_TYPE* paramDest) {
        mutex.claimAtStart();

        if (waiting != NULL) {
            //Writer waiting:
            *paramDest = *src;
            Process* wasWaiting = waiting;
            waiting = NULL; src = NULL;
            freeProcessNoAlt(wasWaiting);
            mutex.release();
        } else {
            //No-one waiting:
            dest = paramDest;
            waiting = currentProcess();
            mutex.release();
            reschedule();
            //Now communication has finished (or we've been poisoned)
            mutex.checkForPoisonAfterComm();
        }
    }

    void beginExtInput(DATA_TYPE* paramDest) {
        mutex.claimAtStart();

        if (waiting != NULL) {
            //Writer waiting:
            *paramDest = *src;
            mutex.release();
        } else {
            //No-one waiting:
            dest = NULL;
            waiting = currentProcess();
            mutex.release();
            reschedule();
            //Now we are in the extended action (or we've been poisoned)
            mutex.claimAtStart();
            *paramDest = *src;
            mutex.release();
        }
    }

    //This method will never throw a poison exception
    void endExtInput() {
        //The reader may have poisoned since beginExtInput
        //If that is the case, we do not want to throw a PoisonException to them
        //The writer can't have poisoned (they are blocked) so it must have been the reader:
        //No need to claim the mutex

        if (false == mutex.checkForPoisonGeneral()) {
            Process* wasWaiting = waiting;
            waiting = NULL; src = NULL;
            freeProcessNoAlt(wasWaiting);
        }
    }

    void output(const DATA_TYPE* paramSrc) {
        mutex.claimAtStart();

        if (waiting != NULL) {
            //Reader waiting:
            if (dest != NULL) {
                //Normal reader:
                *dest = *paramSrc;
                Process* wasWaiting = waiting;
                waiting = NULL; dest = NULL;
                freeProcessNoAlt(wasWaiting);
                mutex.release();
            } else {
                //Alting/extended reader:
                Process* wasWaiting = waiting;
                waiting = currentProcess();
                src = paramSrc;
                freeProcessAlting(wasWaiting);
                mutex.release();
                reschedule();
                //Now communication has finished (or we've been poisoned)
                mutex.checkForPoisonAfterComm();
            }
        } else {
            //No-one waiting:
            src = paramSrc;
            waiting = currentProcess();
            mutex.release();
            reschedule();
            //Now communication has finished (or we've been poisoned)
            mutex.checkForPoisonAfterComm();
        }
    }

    void poisonIn() {
        if (mutex.claimNoPoison())
        {
            Process* wasWaiting = waiting;
            waiting = NULL;
            
            if (wasWaiting != NULL) {
                //They are mid-communication:
                mutex.poisonDuringComm();
            } else {
                //No-one waiting:
                mutex.poison();
            }
            
            if (wasWaiting != NULL) {
                //Might be alting, might not:
                freeProcessAlting(wasWaiting);
            }
        }
    }

    void poisonOut() {
        poisonIn();
    }
    
    //Guard methods:

    bool enable(Process* proc) {
        if (false == mutex.claimNoPoison()) {
            //Channel is poisoned:
            return true;
        } else if (waiting != NULL) {
            //Someone is ready to write
            mutex.release();
            return true;
        } else {
            //Put ourselves in the channel:
            waiting = proc;
            dest = NULL;
            mutex.release();
            return false;
        }
    }

    bool disable(Process* proc) {
        if (false == mutex.claimNoPoison()) {
            //Channel is poisoned:
            return true;
        } else if (waiting != NULL && waiting != proc) {
            //Someone is in the channel and its not us:
            mutex.release();
            return true;
        } else {
            //Only us in the channel, remove ourselves:
            waiting = NullProcessPtr;
            mutex.release();
            return false;
        }
    }    
\end{verbatim}}

\section{Proposed New Algorithm -- Completion State Variables}


{\small\begin{verbatim}
    Process* volatile waiting;
    union {
        const DATA_TYPE* volatile src;
        DATA_TYPE* volatile dest;
    };
    volatile bool isPoisoned;
    bool volatile * volatile commFinished;
    Mutex mutex;

    void checkPoison() {
        if (isPoisoned) {
            mutex.release();
            throw PoisonException();
        }
    }

    void input(DATA_TYPE* paramDest) {
        mutex.claim();
        checkPoison();        

        if (waiting != NULL) {
            //Writer waiting:
            *paramDest = *src;
            Process* wasWaiting = waiting;
            waiting = NULL; src = NULL;
            *commFinished = true;
            freeProcessNoAlt(wasWaiting);
            mutex.release();
        } else {
            //No-one waiting:
            dest = paramDest;
            waiting = currentProcess();
            volatile bool finished = false;
            commFinished = &finished;
            mutex.release();
            reschedule();
            //Now communication has finished (or we've been poisoned)
            if (false == finished)
                throw PoisonException();
        }
    }

    void beginExtInput(DATA_TYPE* paramDest) {
        mutex.claim();
        checkPoison();

        if (waiting != NULL) {
            //Writer waiting:
            *paramDest = *src;
            mutex.release();
        } else {
            //No-one waiting:
            dest = NULL;
            waiting = currentProcess();
            volatile bool finished = false;
            commFinished = &finished;
            mutex.release();
            reschedule();
            //Now we are in the extended action (or we've been poisoned)
            if (false == finished)
                throw PoisonException();
            mutex.claim();
            *paramDest = *src;
            mutex.release();
        }
    }

    //This method will never throw a poison exception
    void endExtInput() {
        //The reader may have poisoned since beginExtInput
        //If that is the case, we do not want to throw a PoisonException to them
        //The writer can't have poisoned (they are blocked) so it must have been the reader:
        //No need to claim the mutex

        if (false == isPoisoned) {
            Process* wasWaiting = waiting;
            waiting = NULL; src = NULL;
            *commFinished = true;
            freeProcessNoAlt(wasWaiting);
        }
    }

    void output(const DATA_TYPE* paramSrc) {
        mutex.claim();
        checkPoison();

        if (waiting != NULL) {
            //Reader waiting:
            if (dest != NULL) {
                //Normal reader:
                *dest = *paramSrc;
                Process* wasWaiting = waiting;
                waiting = NULL; dest = NULL;
                *commFinished = true;
                freeProcessNoAlt(wasWaiting);
                mutex.release();
            } else {
                //Alting/extended reader:
                Process* wasWaiting = waiting;
                waiting = currentProcess();
                src = paramSrc;
                *commFinished = true;
                volatile bool finished = false;
                commFinished = &finished;
                freeProcessAlting(wasWaiting);
                mutex.release();
                reschedule();
                //Now communication has finished (or we've been poisoned)
                if (false == finished)
                    throw PoisonException();
            }
        } else {
            //No-one waiting:
            src = paramSrc;
            waiting = currentProcess();
            finished = false;
            commFinished = &finished;
            mutex.release();
            reschedule();
            //Now communication has finished (or we've been poisoned)
            if (false == finished)
                throw PoisonException();
        }
    }

    void poisonIn() {
        mutex.claim();
        isPoisoned = true;
        //Don't change commFinished
        
        Process* wasWaiting = waiting;
        waiting = NULL;

        if (wasWaiting != NULL) {
            //Might be alting, might not:
            freeProcessAlting(wasWaiting);
        }

        mutex.release();

    }

    void poisonOut() {
        poisonIn();
    }
    
    //Guard variable:
    volatile bool finished;

    //Guard methods:
    
    bool enable(Process* proc) {
        mutex.claim();

        if (isPoisoned) {
            mutex.release();
            return true;
        } else if (waiting != NULL) {
            //Someone is ready to write
            mutex.release();
            return true;
        } else {
            //Put ourselves in the channel:
            waiting = proc;
            dest = NULL;
            //Mainly to stop the pointer being invalid:
            commFinished = &finished;
            mutex.release();
            return false;
        }
    }

    bool disable(Process* proc) {
        mutex.claim();

        if (isPoisoned) {
            mutex.release();
            return true;
        } else if (waiting != NULL && waiting != proc) {
            //Someone is in the channel and its not us:
            mutex.release();
            return true;
        } else {
            //Only us in the channel, remove ourselves:
            waiting = NullProcessPtr;
            mutex.release();
            return false;
        }
    }
\end{verbatim}}


\end{document}
